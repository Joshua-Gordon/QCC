\documentclass[12pt]{book}
\usepackage[utf8]{inputenc}
\usepackage{amsmath,amsfonts,amssymb}

\newcommand{\yup}{\checkmark}
\newcommand{\nup}{$\times$}


\title{Quantum Programming Without Tears}
\author{
Max Cutugno 
\and
Josh Gordon
\and
Tino Tamon
}
\date{\today}

\begin{document}
\maketitle
\tableofcontents


%%%%%%%%%%%%%%%%%%%%%%%%%%%%%%%%%%%%%%%%%%%%%%%%%%%%%%%%%%%%%%%%%%%%%%%%%%%%%%%%%%%%%%%

\chapter{Python based platforms}

Pyquil and QISkit are based on python. In what follows, we describe some preliminary setup
options for running the quantum python programs for these two platforms.

In what follows, we describe an installation on Ubuntu 14.04 running on a 64-bit machine.
The installations on Windows 10 and Mac OS X are similar (and somewhat easier).


%%%%%%%%%%%%%%%%%%%%%%%%%%%%%%%%%%%%%%%%%%%%%%%%%%%%%%%%%%%%%%%%%%%%%%%%%%%%%%%%%%%%%%%

\section{Anaconda}

To install python in a self-contained environment (which may avoid subtle clashes between python2 
and python3 as well as their respective libraries), one option is to use anaconda. 
We stress that this is optional as one can install pyquil directory without using anaconda.

First, we install anaconda into some directory, say {\tt anaconda2}.
Then, to make it visible globally, we update the bash file {\tt .bashrc}:
\begin{quote}
\begin{verbatim}
#Place the following line in the file .bashrc (at the end)
export PATH="$USER/anaconda2/bin:$PATH"
\end{verbatim}
\end{quote}
Here the variable {\tt \$USER} specifies the path to the user's main directory.
Make sure to source the file from the command line:
\begin{quote}
\begin{verbatim}
$ source .bashrc
\end{verbatim}
\end{quote}
Now conda should be visible from anywhere.

We can update conda first (to catch any latest upgrades):
\begin{quote}
\begin{verbatim}
$ conda update conda
\end{verbatim}
\end{quote}

Now, we create a conda {\em environment} for running our python based programs:
\begin{quote}
\begin{verbatim}
# create a new conda environment called 'myquil'
$ conda create --name myquil python=python3.6
$ conda env list
\end{verbatim}
\end{quote}
To enter the conda environment, we do
\begin{quote}
\begin{verbatim} 
$ source activate myquil
(myquil) $
\end{verbatim}
\end{quote}
and to exit we do
\begin{quote}
\begin{verbatim} 
(myquil) $ source deactivate
$
\end{verbatim}
\end{quote}


%%%%%%%%%%%%%%%%%%%%%%%%%%%%%%%%%%%%%%%%%%%%%%%%%%%%%%%%%%%%%%%%%%%%%%%%%%%%%%%%%%%%%%%

\section{Jupyter Notebook}

If one wishes to use Jupyter notebook, we need to the add the conda environment
to the notebook. First, we enter the conda environment and then install a python kernel 
for jupyter notebook and connect it to the conda environment:
\begin{quote}
\begin{verbatim} 
(myquil) $ pip install ipykernel
(myquil) $ python -m ipykernel install --user --name myquil
\end{verbatim}
\end{quote}
%Installed kernelspec myquil in <$USER>/.local/share/jupyter/kernels/myquil
Now, we can run jupyter notebook (from outside conda) and choose the appropriate 
kernel (and hence conda environment).



\chapter{Pyquil}

On Linux, we can install pyquil directly (using pip). 
The installations on Windows 10 and MacOS X are similar (we might need brew on MacOS X
instead of pip).

\section{Anaconda}

But, we can also install it within a conda environment as follows:
\begin{quote}
\begin{verbatim} 
$ source activate myquil
(myquil) $ pip install pyquil
(myquil) $ pyquil-config-setup
Welcome to PyQuil!
Enter the required information below for Forest connections.
If you haven't signed up yet you will need to do so first 
  at https://forest.rigetti.com
Forest API Key: <ENTER API HERE>
User ID: <ENTER ID HERE>
Pyquil config file created at '/home/tino/.pyquil_config'
If you experience any problems see the guide 
  at https://go.rigetti.com/getting-started
\end{verbatim}
\end{quote}
Now, we can run some python programs in a directory of examples:
\begin{quote}
\begin{verbatim} 
(myquil) $ python run_quil.py hello_world.quil 
Running Quil Program from:  hello_world.quil
---------------------------
Output: 
[[1, 0, 0, 0, 0, 0, 0, 0]]
\end{verbatim}
\end{quote}


%%%%%%%%%%%%%%%%%%%%%%%%%%%%%%%%%%%%%%%%%%%%%%%%%%%%%%%%%%%%%%%%%%%%%%%%%%%%%%%%%%%%%%%

\chapter{QISkit}


%%%%%%%%%%%%%%%%%%%%%%%%%%%%%%%%%%%%%%%%%%%%%%%%%%%%%%%%%%%%%%%%%%%%%%%%%%%%%%%%%%%%%%%

\chapter{Quipper}

This seems to be the hardest to install.
QLSA purportedly is implemented here.


%%%%%%%%%%%%%%%%%%%%%%%%%%%%%%%%%%%%%%%%%%%%%%%%%%%%%%%%%%%%%%%%%%%%%%%%%%%%%%%%%%%%%%%

\chapter{Liquid}

This seems to be the most straightforward to install.
It is unclear if QLSA runs correctly here.


%%%%%%%%%%%%%%%%%%%%%%%%%%%%%%%%%%%%%%%%%%%%%%%%%%%%%%%%%%%%%%%%%%%%%%%%%%%%%%%%%%%%%%%

\chapter{Sanity Checks}


\section{Main algorithms}

We examine several basic known quantum algorithms and verify if they are implemented 
within a certain quantum software platform.

\bigskip

\begin{tabular}{|l||c|c|c|c||} \hline
NAME				& Pyquil& QISkit& Quipper& Liquid \\ \hline \hline
Bell circuit 		& \yup	& \yup	& \yup	& \yup \\ \hline
teleportation 		& \yup	& \yup	& \yup	& \yup \\ \hline
Deutsch	 			& \nup	& \nup	& \nup	& \nup \\ \hline
Deutsch-Jozsa		& \nup	& \nup	& \nup	& \nup \\ \hline
Bernstein-Vazirani	& \nup	& \nup	& \nup	& \nup \\ \hline
Simon				& \nup	& \nup	& \nup	& \nup \\ \hline
QFT					& \nup	& \nup	& \yup	& \yup \\ \hline
Phase estimation	& \nup	& \nup	& \nup	& \nup \\ \hline
Shor				& \nup	& \nup	& \nup	& \nup \\ \hline
Grover				& \nup	& \nup	& \nup	& \nup \\ \hline
Singular value estimation	& \nup	& \nup	& \nup	& \nup \\ \hline
Hamiltonian simulation	& \nup	& \nup	& \nup	& \nup \\ \hline
HHL					& \nup	& \nup	& \nup	& \nup \\ \hline
\end{tabular}

\bigskip


\section{Additional algorithms}

\begin{enumerate}
\item (D\"{u}rr and H$\o{}$yer) Finding the minimum element in an array.

\item (Ambainis) Determining if an array contains distinct items.

\item (Child and Goldstone) Continuous-time spatial search.

\end{enumerate}


%%%%%%%%%%%%%%%%%%%%%%%%%%%%%%%%%%%%%%%%%%%%%%%%%%%%%%%%%%%%%%%%%%%%%%%%%%%%%%%%%%%%%%%

\chapter{Graphical quantum circuit tools}



%%%%%%%%%%%%%%%%%%%%%%%%%%%%%%%%%%%%%%%%%%%%%%%%%%%%%%%%%%%%%%%%%%%%%%%%%%%%%%%%%%%%%%%

\chapter{Cross compilation}


%%%%%%%%%%%%%%%%%%%%%%%%%%%%%%%%%%%%%%%%%%%%%%%%%%%%%%%%%%%%%%%%%%%%%%%%%%%%%%%%%%%%%%%

\chapter{Baffling questions}

\begin{enumerate}
\item Why does Quipper provide a {\em cloning} function?

\item In Liquid, is F\# now obsolete and being replaced by Q\#? If so, why?
\end{enumerate}


%%%%%%%%%%%%%%%%%%%%%%%%%%%%%%%%%%%%%%%%%%%%%%%%%%%%%%%%%%%%%%%%%%%%%%%%%%%%%%%%%%%%%%%



%%%%%%%%%%%%%%%%%%%%%%%%%%%%%%%%%%%%%%%%%%%%%%%%%%%%%%%%%%%%%%%%%%%%%%%%%%%%%%%%%%%%%%%

\end{document}



%%%%%%%%%%%%%%%%%%%%%%%%%%%%%%%%%%%%%%%%%%%%%%%%%%%%%%%%%%%%%%%%%%%%%%%%%%%%%%%%%%%%%%%
% RIGETTI QUIL
Welcome to PyQuil!
Enter the required information below for Forest connections.
If you haven't signed up yet you will need to do so first at https://forest.rigetti.com
Forest API Key: nmRPAVunQl19TtQz9eMd11iiIsArtUDTaEnsSV6u
User ID: 9db973f6-2c27-4c97-921c-fb3039c443a8
Pyquil config file created at '/home/tino/.pyquil_config'
If you experience any problems see the guide at https://go.rigetti.com/getting-started
%%%%%%%%%%%%%%%%%%%%%%%%%%%%%%%%%%%%%%%%%%%%%%%%%%%%%%%%%%%%%%%%%%%%%%%%%%%%%%%%%%%%%%%

